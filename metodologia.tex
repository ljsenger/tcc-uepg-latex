\chapter{Materiais e Métodos}

\section{Considerações Iniciais}
A partir da classe com menor representatividade na base de dados, classe 4 - Algodão Americano/Salgueiro com 2.747 observações, sendo todos os dados selecionados, foram extraídas das outras seis classes de tipos de cobertura florestal 2.747 observações, selecionadas aleatoriamente com o auxilio do software R, totalizando o conjunto de teste com 19.229 observações, conforme a tabela \ref{tb:dados}.

\begin{table}[htbp]
\caption{Observações Selecionadas}
\label{tb:dados}
\centering
\setlength{\tabcolsep}{5pt}
\begin{tabular}{cccccc}
\hline
Tipo de Cobertura  &Total de  &Observações  &Porcentagem por \\
Florestal &Observações &Selecionadas &Tipo de Cobertura \\
\hline
Classe 1 &211.840 &2747 &1,29\% \\
Classe 2 &283.301 &2747 &0,97\% \\
Classe 3 &35.754  &2747 &7,68\% \\
Classe 4 &2.747   &2747 &100\% \\
Classe 5 &9.493   &2747 &28,94\% \\
Classe 6 &17.367  &2747 &15,82\% \\
Classe 7 &20.510  &2747 &13,39\% \\
\hline
\textbf{Total} &\textbf{581.012} &\textbf{19.229} &\textbf{3,31\%} \\
\hline
\end{tabular}
\\
%\singlespacing
%\text{\footnotesize Fonte: O autor}
\end{table}
